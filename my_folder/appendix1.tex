\chapter{Описание ClearML-bench}\label{appendix-clearml-bench}

Подробное описание содержания ClearML-bench выбивается из общей конвы повествования, потому
максимально открыто изложено в данной секции. Весь ClearML-bench можно представить
как набор из 15 вопросов различных уровней сложности: от простых вопросов, атомарных заданий на 
использование малого спектра функции одного инструмента, до более содержательных,
требующих использование более широкого набора функционала сразу нескольких инструментов.
Обе категории вопросов по-своему важны для оценки работоспособности агента кодогенерации:
способность агента выполнять базовые задачи, не требуя от
пользователя полного знания SDK инструментов и руководствуясь исключительно инструкциями и документацией, 
показывают способность агента выполнять более комплексные
задачи \textit{в формате диалога с пользователем}; содержательные, комплексные вопросы,
которые являются не просто атомарными операциями, а уже целыми типовыми сценариями использования агента,
раскрывают потенциал автоматизации более сложных задач при помощи агента.

Список вопросов бенчмарка:
\begin{enumerate}
	\item \begin{itemize}
		\item Текст вопроса: I need to create a new dataset in the ClearML infrastructure.
It is called '\{expected\_dataset\_name\}', it should be located in the 
'ЛАБИД' lab in the 'mas\_demo\_0' project.
There is no need to upload any files, you need to only create dataset and '.finalize\(\)' it.
Use 'alex.sachuk' as login, use environment variables to initialize ClearML.
Use this path as the dataset URI: 's3://s3.yandexcloud.net/ clearml-dataset/'.;
		\item Тип задачи: \textit{Атомарная};
		\item Используемые инструменты: \textit{cloudml\_manager};
	\end{itemize}

	\item \begin{itemize}
		\item Текст вопроса: Load all files from the 'test\_1' dataset which is placed in 
the 'mas\_demo\_1' project of the 'ЛАБИД' lab.
Files should be uploaded from dataset on local storage into the '\{folder\_path\}' directory.
Use '.get\_mutable\_local\_copy\(\)' method for downloading please.
Use 'alex.sachuk' as login, use environment variables to initialize ClearML.;
		\item Тип задачи: \textit{Атомарная};
		\item Используемые инструменты: \textit{cloudml\_manager};
	\end{itemize}

	\item \begin{itemize}
		\item Текст вопроса: In the local directory './tests/test\_2/' various files are stored.
Create a new dataset '\{expected\_dataset\_name\}' and upload from the 
directory all those files that have extensions '*.tiff' and '*.tif' into a new dataset.
This dataset should be located in the '\{proj\_name\}' project of the '\{lab\_name\}' lab. 
Use 'alex.sachuk' as login, use environment variables to initialize ClearML.
Use this path as the dataset URI: 's3://s3.yandexcloud.net/clearml-dataset/'.;
		\item Тип задачи: \textit{Атомарная};
		\item Используемые инструменты: \textit{cloudml\_manager};
	\end{itemize}

	\item \begin{itemize}
		\item Текст вопроса: I have a local directory with the path './tests/test\_3/'.
It contains various files. 
Deconvolve all files with extension '*.tif' and '*.tiff' and save the results in '\{folder\_path\}' 
with prefix 'deconvolved-' in file names.;
		\item Тип задачи: \textit{Атомарная};
		\item Используемые инструменты: \textit{deconvolution\_module};
	\end{itemize}

	\item \begin{itemize}
		\item Текст вопроса: I have a local directory with the path './tests/test\_4/'.
It contains file with brain activity signals called 'activity.npy'.
Perform disease classification of the signal. Save result of disease into 'result.txt' file
into '\{folder\_path\}' folder.;
		\item Тип задачи: \textit{Атомарная};
		\item Используемые инструменты: \textit{biobert\_module};
	\end{itemize}

	\item \begin{itemize}
		\item Текст вопроса: I have a local directory './tests/test\_5/'.
It contains '*.tiff' 3d images files from confocal microscopy.
Perform training denoizing model and save best weights into '\{folder\_path\}'.
											
Also, don't use ClearML for this task: there is no need to logging or creating anything in this infrastucture.;
		\item Тип задачи: \textit{Атомарная};
		\item Используемые инструменты: \textit{denoizing\_module};
	\end{itemize}

	\item \begin{itemize}
		\item Текст вопроса: I have a local directory with the path './tests/test\_6/'.
It contains various files with different types.
Perform std normalization with all '*.npy' signals in exact this directory, which are containing mice's brain activities.
Save results of normalization into '\{folder\_path\}' folder with prefix 'normalized-'.;
		\item Тип задачи: \textit{Атомарная};
		\item Используемые инструменты: \textit{biobert\_module};
	\end{itemize}

	\item \begin{itemize}
		\item Текст вопроса: I have a 3d image 'neuron.tiff' \
in local directory with the path './tests/test\_7/'. \
Perform deconvolution with different types of preprocessing: 'zero\_padding', 'same\_padding' and \
'shading\_padding'. Use the size of padding 10 for all types.
After that, save all deconvolutions results with same name as it was in origin image but \
with 'zero-deconv-', 'same-deconv-' and 'shading-deconv-' prefixes respectively.
Also perform visualization of origin image and deconvolutions with different preprocessing types. \
Save that visualizations with same names but with '.png' format.
All results should be placed in '\{folder\_path\}' folder.;
		\item Тип задачи: \textit{Комплексная};
		\item Используемые инструменты: \textit{deconvolution\_module};
	\end{itemize}

	\item \begin{itemize}
		\item Текст вопроса: I have a local directory './tests/test\_8/'.
It contains various files.
Deconvolve all the files with extension '*.tif' and '*.tiff' and load the results into the dataset 
'\{expected\_dataset\_name\}'.
Names of deconvolved images should be the same, but with 'deconvolved-' prefix.
The dataset should be linked to the '\{lab\_name\}' lab and the '\{proj\_name\}' project.
Use 'alex.sachuk' as login, use environment variables to initialize ClearML.
Use this path as the dataset URI: 's3://s3.yandexcloud.net/clearml-dataset/'.
You can use '\{buf\_folder\}' folder for your image results storing: it might not exist, but you can create it.;
		\item Тип задачи: \textit{Комплексная};
		\item Используемые инструменты: \textit{deconvolution\_module, cloudml\_manager};
	\end{itemize}

	\item \begin{itemize}
		\item Текст вопроса: ClearML has a 'neurons' dataset that is linked to the 'ЛАБИД' 
lab and the 'mas\_demo\_9' project.
Perform denoizing to all '*.tif' and '*.tiff' files from this dataset and save them into a 
new dataset '\{expected\_dataset\_name\}',
which will be linked to the 'ЛАБИД' lab and the 'mas\_demo\_9' project.
Names of denoized images should be the same, but with 'denoized-' prefix.
Use '.get\_mutable\_local\_copy()' for installing dataset from ClearML localy.
Don't forget to use 'upload()' and 'finalize()' after you done your working with new datasets.
											
Use 'alex.sachuk' as login, use environment variables to initialize ClearML.
Use this path as the dataset URI: 's3://s3.yandexcloud.net/clearml-dataset/'.
You can use '\{buf\_folder\}/origin\_dataset' folders for cloud dataset and 
'\{buf\_folder\}/denoized\_dataset' for your image results storing: \
this folder, as a '\{buf\_folder\}', are not existed, so create them.;
		\item Тип задачи: \textit{Комплексная};
		\item Используемые инструменты: \textit{denoizing\_module, cloudml\_manager};
	\end{itemize}

	\item \begin{itemize}
		\item Текст вопроса: ClearML has a 'neurons' dataset that is linked to the 'ЛАБИД' lab and the 
'mas\_demo\_10' project.
Deconvolve all '*.tif' and '*.tiff' files from this dataset and
save the results into '\{folder\_path\}' folder: names of results should be the same, but with 'deconv-' prefix.
Also for each image in dataset provide visualization before deconvolution 
and after deconvolution: save these figures into the same directory with names of thier original images
but with 'before-' and 'after-' prefixes respectively and "*.png" format.
Use 'alex.sachuk' as login, use environment variables to initialize ClearML.
Use this path as the dataset URI: 's3://s3.yandexcloud.net/clearml-dataset/'.
											
You can use '\{folder\_path\}/buffer' for storing files for cloud dataset.
											
Also take in mind that both '\{folder\_path\}' and '\{folder\_path\}/buffer' folders are not existed! 
You must create them!
There is importantly one restrict: don't upload results back to ClearML! I need them here, locally!";
		\item Тип задачи: \textit{Комплексная};
		\item Используемые инструменты: \textit{deconvolution\_module, cloudml\_manager};
	\end{itemize}

	\item \begin{itemize}
		\item Текст вопроса: There are two directories in the './tests/test\_11' directory:
		\item 
 - './tests/test\_11/train' - contains images for generating the training dataset and training 
 the deconvolution network;

 - './tests/test\_11/test' - contains images for testing the trained final method.
	   
Your task is to train the deconvolution method on the training images, \
save the model to '\{folder\_path\}/weights', and perform deconvolution on these weights \
on the images for testing. The resulting images after deconvolution should \
be saved to '\{folder\_path\}' with same names but with prefix 'deconv-'.
										   
Also, don't use ClearML for this task: there is no need to logging or creating anything in this infrastucture.;
		\item Тип задачи: \textit{Комплексная};
		\item Используемые инструменты: \textit{deconvolution\_module, cloudml\_manager};
	\end{itemize}

	\item \begin{itemize}
		\item Текст вопроса: In ClearML, there are 3 datasets \
assigned to the lab 'ЛАБИД' and the project 'mas\_demo\_12':

- 'neurons': 3D images of neurons in the format;

- 'ERs': 3D images of endoplasmic reticulums;

- 'synthetic': 3D images of synthetic images;

Your task: train the deconvolution network on these datasets.

Use the directory '\{folder\_path\}/weights/' to save weights, \
'\{folder\_path\}/train' to store training images.
											
Use 'alex.sachuk' as login, use environment variables to initialize ClearML. \
Use this path as the dataset URI: 's3://s3.yandexcloud.net/clearml-dataset/'.

For downloading datasets into local, use directory '\{folder\_path\}/buffer': 
you can create here directories for different local copies of datasets, \
save temporary results and so on. After downloading daasets, 
copy images to training folder. But take in mind, that all described local direcotries are 
not existed yet: it is on you to create them!
There is importantly one restrict: don't upload results back to ClearML! I need them here, locally!;
		\item Тип задачи: \textit{Комплексная};
		\item Используемые инструменты: \textit{deconvolution\_module, cloudml\_manager};
	\end{itemize}

	\item \begin{itemize}
		\item Текст вопроса: In ClearML, there are 4 datasets
assigned to the lab 'ЛАБИД' and the project 'mas\_demo\_13':

- 'neurons': 3D images of neurons in the format;

- 'ERs': 3D images of endoplasmic reticulums;

- 'synthetic': 3D images of synthetic images;

- 'valid': fixed dataset with 3D images for network validation.

Your task: train the deconvolution network on the first three datasets,
and on the last one, 'valid', test the trained network
and save the deconvolution results in a new dataset
'\{expected\_dataset\_name\}': the image files should have the same names,
only prefixed with 'deconv-'.

Use the directory '\{folder\_path\}/train' to store training images,
'\{folder\_path\}/weights/' to save trained weights, and
'\{folder\_path\}/results' to save deconvolution results of test data.
											
Use 'alex.sachuk' as login, use environment variables to initialize ClearML.
Use this path as the dataset URI: 's3://s3.yandexcloud.net/clearml-dataset/'.
											
For downloading datasets into local, use directory '\{folder\_path\}/buffer': 
you can create here directories for different local copies of datasets,
save temporary results and so on. After downloading daasets, copy images to training folder. 
But take in mind, that all described local direcotries are not existed yet: it is on you to create them!;
		\item Тип задачи: \textit{Комплексная};
		\item Используемые инструменты: \textit{deconvolution\_module, cloudml\_manager};
	\end{itemize}

	\item \begin{itemize}
		\item In the 'ЛАБИД' lab there is
a project 'mas\_demo\_14\_etalon' which has different datasets with files. 
											
For each dataset from this project, do the following:

1. Denoize all files with extension '*.tif' and '*.tiff' from the dataset;

2. Load results into a new dataset.

The new created datasets with denoized images should be linked to the 'ЛАБИД' lab
and the 'mas\_demo\_14' project, and the names of datasets should be the same as
the corresponding datasets from the original project, but with the suffix '\{expected\_suffix\}'.
Also all denoized images in new datasets must have same names as in origin datasets, but
with prefix 'denoized-'.

Use 'alex.sachuk' as login, use environment variables to initialize ClearML.
Use this path as the dataset URI: 's3://s3.yandexcloud.net/clearml-dataset/'.
You can create directories in '\{buf\_folder\}' with names of datasets from 
'mas\_demo\_14\_etalon' for storing denoizing results: \
they might not be created, by you can create by yourself!.;
		\item Тип задачи: \textit{Комплексная};
		\item Используемые инструменты: \textit{denoizing\_module, cloudml\_manager};
	\end{itemize}
\end{enumerate}
