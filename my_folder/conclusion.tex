\chapter*{Заключение} \label{ch-conclusion}
\addcontentsline{toc}{chapter}{Заключение}	% в оглавление 

В ходе проделанной работы была реализована мультиагентная система, способная работать с различными 
форматами инструментов, что позволяет получать преимущества от каждых из инструментов: безопасность при
использовании JSON-вызовов и инструментов по MCP-интерфейсам и гибкость логики, поддержку мультимодальности
при вызове этапов кодогенерации.

Кроме того, вынос агента кодогенерации в отдельного агента также значительно улучшил применимость системы
во внутренних задачах лаборатории: этап подвергся значительному числу улучшений, которые в исходном решении
CodeAct были бы тяжело реализуемы. К этим улучшениям относятся и рефлексии в случае ошибок, и подключение
различных ретриверов для решения промежуточых задач, и сохранение кодогенераций для некоторых типовых задач
в Jupyter Notebook. Такое решение позволило не только применять мультиагентную систему в решении 
\textit{внутренних} задач, но и делать это \textit{оптимально}: на представленных данных расход токенов
был уменьшен на 33\%, притом точность решения была не хуже (а даже немного лучше) при решении в сравнении
с частоиспользуемыми вариантами ретриверов.

Также важно выделить различные качественные свойства системы, а именно поддержка анализа мультимодальности,
возможность исользовать модели, как развернутые локально, так и работающих через API запросы,
а также сохранение кодогенераций в Jupyter Notebook.
Действительно, в отличие от аналога, данное решение позволяет работать с мультимодальными данными
за счет гибкости реализцаии нужно для анализа разных модальностей интерфейса: разработка
позволяет использовать различные появляющиеся большие фундаментальные модели модели, посвященные анализу
различных специфичных данных. 

Свойство подключать разные (M)LLM-модели для принятий решений позволяет не только выбирать между необходимыми
политиками и свойствами моделей, но и позволяет выбирать среди конфиденциальностью и портативностью решений:
выбор использовать модели на локальных мощностях позволяет избегать лишних рисков с утечкой приватных данных,
которые являются в некотором роде ``валютой'' исследовательских групп, а выбор использовать удаленные решения
позволяет экономить на энергопотреблении и приобретением высокопроизводительных компьютеров и вычислительных 
устройств.

Однако, данное решение может содержать различные недостатки: на сегодняшний день одной из наиболее беспокоящих
ученых тем является уровень достоверности вывода (М)LLM-моделей. Каждый день выходят новые, бросающие вызов
моделям и их создателям бенчмарки, предназначенные для оценки способностей моделей к решению узких, но очень
важных задач, с которыми модели не очень отлично справляются. Данное свойство не запрещает использовать 
разработанную систему как асистента в исследованиях, который способен быстро и точно производить доступ
к различным данным, моделям, а также выполнять разные кодовые задачи, однако не позволяет использовать в 
качестве асистента для генерации новых знаний и выводов. Стоит отметить, что понимание данной ситуации в 
современных исследованиях и позволило сместить фокус с целей данной работы, потому наличие данного вопроса
не является узким местом в предлагаемом решении.     

Выражаю благодарность своему научному руководителю за визионерскую и профессиональную помощь
в реализации предлагаемой работы. Выражаю также особую благодарность консультанту от 
Лаборатории Анализа Биомедицинских Изображений и Данных (ЛАБИД).
