\chapter*{Введение} % * не проставляет номер
\addcontentsline{toc}{chapter}{Введение} % вносим в содержание

Первая половина 20-ых годов в мире машинного обучения уже ознаменовалось 
как веха больших фундаментальных моделей.
Наиболее почетное место среди них занимают генеративные текстовые модели: 
с выходом моделей BERT и семейства GPT они значительно повлияли не только на 
исследования в области машинного обучения, но и на привычную бытовую жизнь, 
технологические процессы как в личных, так и в индустриальных подходах. 
Не обошло стороной это веяние и исследовательсие организации - 
действительно, способность моделей иметь некоторое, хоть местами и поверхностное, 
но представление о разных областях уже позволяет вовлекать GenAI в научные процессы.

Этому свидетельствует значительное число работ, 
посвященные интеграциям больших языковых моделей (или БЯМ, LLM) 
в различные виды деятельности ученых. Уже сейчас существуют
направления исследований, направленных на \textit{автоматизацию и упрощение различных процессов}
лабораторий:
\begin{itemize}
	\item автоматизация поиска статей и их анализа;
	\item консультация в формате диалога по поиску решений на вопросы 
узкоспециализированного или междисциплинарного профиля;
	\item развитие понимания узкоспециализированных областей;
	\item автоматическая генерация корректного научного текста;
	\item автоматизация поиска оптимального решения задач при помощи кодовых генераций; 
\end{itemize}
Стоит отметить, что присутствуют и работы в области создания систем автоматизации полного 
исследовательского цикла, но они носят узкоспециализированный характер: они создаются только
в рамках исследований методов машинного обучения и их непосредсвенных приложений.

Исследование и развитие автоматизаций процессов исследовательских групп имеет 
большую актуальность: автоматизация различных процессов, которые требуют 
наименьшей дополнительной валидации или не требуют её вообще, позволяют быстрее 
выполнять исследовательские задачи. Кроме того, автоматизация невозможна без обучения
и снабжения LLM специфическими знаниями - это может привести к галлюцинациям и неожиданным
в плохом смысле слова результатам. 

Куда более обделенным направлением работ в исследовании автоматизаций является 
исследование интеграции БЯМ во внутренние ресурсы и процессы лабораторий, 
а также практическая разработка и тестирование таких систем. 
Действительно, интеграция БЯМ во внутренние процессы сталкивает со следующими проблемами:
\begin{itemize}
	\item Отсутствие единых стандартов работ в лабораторий: хотя многие молодые лаборатории
стараются перенимать принципы организаций с индустрии, большинство исследовательских групп
имеют собственное, зачастую непопулярное видение организаций различных процессов 
(хранения данных, логирование экспериментов и т.д), которые могут быть незнакомы БЯМ. Это
приводит к малой универсальности предлагаемых решений;
	\item Наличие внутренних разработок: многие лаборатории используют свои внутренние 
специфические инструменты, библиотеки и фреймворки, информация о которых могла 
быть или не представлена в обучающих выборках LLM, или быть малорепрезентативна;
	\item Мультимодальность объектов исследования: не смотря на то, что объекты исследований
лабораторий могут иметь различную модальность, создатели и исследователи 
систем автоматизаций на базе LLM предлагают решения, работающие исключительно с 
текстами, не оставляя в системах места для мультимодальных данных.  
\end{itemize}

Целью данной работы ставится разработка системы с пользовательским интерфейсом, которая 
решает автоматизирует процессы пользователя, поддерживает различные форматы инструментов 
лабораторий, а также способна поддерживать интеграцию процессов работ с данными разнличной
модальности.

\newpage
