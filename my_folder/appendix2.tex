\chapter{Используемые инструменты}\label{appendix-tools}							% Заголовок

Описание всех доступных инструментов стоит разбить на две части: инструменты агента 
кодогенерации ExCodeAgent-MM и всей MAS ExCodeAct.

\section{Список инструментов апробации ExCodeAgent-MM}\label{appendix-tools:sec1}

Полный список инструментов с их англоязычным ``кратким'' описанием, составленным пользователем для 
Coarse фильтрации:
\begin{enumerate}
	\item ``cloudml\_manager'' - ``A module for working with ClearML infrastructure: 
to manage connections, working with cloud data in datasets, logging models and etc.'';
	\item ``deconvolution\_module'' - ``Class which provides all requirement functions for DL deconvolution tasks.'';
	\item ``denoizing\_module'' - ``Class which provides functions for noise reduction from 3D images 
(aka deconvolution algorithms).'';
	\item ``biobert\_module'' - ``Module for working with brain activity signals.'';
	\item ``spinetool\_module'' - ``Module with different methods for working with dendritic spines.''.
\end{enumerate}

Основные классы и методы 'cloudml\_manager', принимающие участие в тестировании ExCodeAgent-MM:
\begin{itemize}
	\item ``LabDataset'' - класс для предоставления доступа к датасетам лабораторных проектов в 
инфраструктуре ClearML. Представляет статические методы для получения экземпляров типа 
``ClearML.Dataset'', через которые можно получать доступ к файлам, а также модифицировать содержание
самих датасетов. Основные методы класса:
	\begin{itemize}
		\item ``def LabDataset.create(...) -> ClearML.Dataset'' - создает новый, пустой датасет в проекте
определенной лаборатории;
		\item ``def LabDataset.get(...) -> ClearML.Dataset'' - возвращает существующий датасет по
имени лаборатории, проекта, и имени самого датасета.
	\end{itemize}

	\item ``InitConnection'' - класс инициализации доступа к инфраструктуре ClearML. Представляет
два метода для инициализации файла конфигурации:
	\begin{itemize}
		\item ``def InitConnection.init\_with\_env\_vars() -> None'' - через уже определенные переменные 
окружения;
	\item ``def InitConnection.init\_with\_python\_vars(...) -> None'' - через значения, определенные
в питоновских переменных;
	\end{itemize}
\end{itemize}

Основные классы и методы 'deconvolution\_module', принимающие участие в тестировании ExCodeAgent-MM:
\begin{enumerate}
	\item ``cloudml\_manager'' - ``A module for working with ClearML infrastructure: 
to manage connections, working with cloud data in datasets, logging models and etc.'';
	\item ``deconvolution\_module'' - ``Class which provides all requirement functions for DL deconvolution tasks.'';
	\item ``denoizing\_module'' - ``Class which provides functions for noise reduction from 3D images 
(aka deconvolution algorithms).'';
	\item ``biobert\_module'' - ``Module for working with brain activity signals.'';
	\item ``spinetool\_module'' - ``Module with different methods for working with dendritic spines.''.
\end{enumerate}

Основные классы и методы 'deconvolution\_module', принимающие участие в тестировании ExCodeAgent-MM:
\begin{itemize}
	\item ``def DeconvolutionModule.load\_3d\_img(...) -> np.ndarray'' - статический метод загрузки
изображений в формате numpy массива;
	\item ``def DeconvolutionModule.make\_deconvolution(...) -> np.ndarray'' - мокап-метод, который 
осуществляет деконволюцию изображения при помощи нейронной сети, опираясь на переданный путь весов 
самой сети;
	\item ``def DeconvolutionModule.train\_model(...) -> bool'' - мокап-метод, который
осуществляет обучение сети на основе файлов, переданных для генерации датасета, и параметров обучения;
	\item ``def DeconvolutionModule.save\_3d\_img(...) -> None'' - статический метод сохранения
трехмерных изображений;
	\item ``def DeconvolutionModule.zero\_padding(...) -> np.ndarray'' - статический метод 
для добавления отступов вдоль оси Z, заполненных нулями;
	\item ``def DeconvolutionModule.shading\_padding(...) -> np.ndarray'' - статический метод 
для добавления отступов вдоль оси Z при помощи повторения крайних слоев с затуханием интенсивности;
	\item ``def DeconvolutionModule.same\_padding(...) -> np.ndarray'' - статический метод для добавления
отступов вдоль оси Z при помощи повторения крайних слоев;
	\item ``def DeconvolutionModule.visualize\_3d\_image(...) -> matplotlib.Figure'' - 
статический метод создания ``изображений-развертки'', демонстрирующее содержимое трехмерного снимка
на плоскости.
\end{itemize}

Основные классы и методы 'denoizing\_module', принимающие участие в тестировании ExCodeAgent-MM:
\begin{itemize}
	\item ``def DenoizingModule.load\_3d\_img(...) -> np.ndarray'' - статический метод загрузки
изображений в формате numpy массива;
	\item ``def DenoizingModule.make\_denoizing(...) -> np.ndarray'' - мокап-метод, который 
осуществляет устранение шумов изображения при помощи нейронной сети, опираясь на переданный путь весов самой сети;
	\item ``def DenoizingModule.train\_model(...) -> np.ndarray'' - мокап-метод, который осуществляет 
обучение сети денойзинга на основе файлов, переданных для генерации датасета, и параметров обучения;
	\item ``def DenoizingModule.save\_3d\_img(path: str) -> np.ndarray'' - статический метод сохранения
трехмерных изображений;
	\item ``def DenoizingModule.visualize\_3d\_image(...) -> matplotlib.Figure'' - 
статический метод создания ``изображений-развертки'', демонстрирующее содержимое трехмерного снимка
на плоскости.
\end{itemize}

Основные классы и методы 'bio\_bert\_module', принимающие участие в тестировании ExCodeAgent-MM:
\begin{itemize}
	\item ``def BioBERT.read\_signal(...) -> np.ndarray'' - статический метод загрузки
сигнала, хранящегося в формате `*.npy' numpy массива;
	\item ``def DenoizingModule.classify\_signal(...) -> str'' - статический мокап-метод классификации
болезни по активности сигнала;
	\item ``def DenoizingModule.std\_normalize\_signals\_features(...) -> np.ndarray'' - статический 
метод для нормализации признаков нейронной активности;
\end{itemize}

Модуль 'spinetool\_module' не использовался напрямую в тестировании ClearML-bench ввиду и без того 
долгого времени работы тестирования, хотя на практике инструмент исследователями применяется в ряде
научных работ. 
Сам модуль сильно не отличается структурно от ранее перечисленных модулей: 
он также состоит из статических методов, имеет методы как преобразующие данные, так и осуществляющие
визаулизацию и анализ посредством различных классических алгоритмов машинного обучения.

Здесь важно отметить следующее: некоторые модули и методы применялись в формате \textit{мокап} 
модулей и методаов. Это значит, что они не выполняют в действительности описанный функционал, 
а выполняют некоторое другое, ожидаемое в процессе тестирование действие.
Такое решение может показаться ``нечестным'' в условиях апробации, но стоит отметить две причины, 
по которой это решение валидное и приемлемое в рамках тестирования:
\begin{itemize}
	\item неопределенность результата: методы, работающие со стохастическими алгоритмами, 
имеют неопределенный конечный результат, валидация и корректность которого представляется трудной. 
Замена истинного алгоритма на некоторый другой иммитирующий алгоритм с детерминированным 
выводом позволяет получить однозначный результат для последующей валидации;
	\item инвариантность для LLM-модели: сама модель не видит то, что происходит внутри функции - она 
осуществляет вывод, опираясь на документацию и сигнатуру функции. Исполняется ли при вызове настоящий
алгоритм или некоторый иммитирующий алгоритм - информация, остающаяся недоступной для LLM. Это позволяет 
проецировать результаты тестирования агента на реальные случаи применения с реальными, истинными 
алгоритмами;
\end{itemize}

\section{Список инструментов ExCodeAct}\label{appendix-tools:sec2}

Полный список прочих, не кодовых инструментов, принимающих участие в тестировании
(например, мультимодальности в \ref{appendix-multimodal}), следующий:
\begin{itemize}
	\item ``arxiv\_search'' - метод поиска научных статей на базе arxiv по запросу;
	\item ``list\_buckets'' - выводит список доступных S3 бакетов;
	\item ``list\_objects'' - выводит список ключей объектов в выбранном S3 бакете;
	\item ``download\_object'' - загружает в указанный путь объект из выбранного S3 бакета. 
\end{itemize}
