%%% Внесите свои данные - Input your data
%%
%%
\newcommand{\Author}{И.О.\,Фамилия} % И.О. Фамилия автора 
\newcommand{\AuthorFull}{Фамилия Имя Отчество} % Фамилия Имя Отчество автора
\newcommand{\AuthorFullDat}{Фамилия Имя Отчество} % Фамилия Имя Отчество автора в дательном падеже (Кому?) 
\newcommand{\Supervisor}{И.О.\,Фамилия} % И. О. Фамилия научного руководителя
\newcommand{\SupervisorJob}{должность} %
\newcommand{\SupervisorDegree}{степень} %
\newcommand{\SupervisorTitle}{звание} % 
%%
%%
\newcommand{\Head}{И.О.\,Фамилия} % И. О. Фамилия руководителя подразделения (руководителя ОП)
\newcommand{\HeadDegree}{Должность руководителя}% Только должность:   
%Руководитель ОП 
%Заведующий кафедрой
%Директор Высшей школы
\newcommand{\HeadDep}{M} % заменить на краткую аббревиатуру подразделения или оставить пустым, если утверждает руководитель ОП
\newcommand{\ConsultantNorm}{И.О.\,Фамилия} % И. О. Фамилия консультанта по нормоконтролю. ТОЛЬКО из числа ППС!
\newcommand{\ConsultantNormDegree}{должность, степень} %   
\newcommand{\ConsultantExtra}{И.О.\,Фамилия} % И. О. Фамилия дополнительного консультанта 
\newcommand{\ConsultantExtraDegree}{должность, степень} % 
\newcommand{\Reviewer}{И.О.\,Фамилия} % И. О. Фамилия резензента. Обязателен только для магистров.
\newcommand{\ReviewerDegree}{должность, степень} % 
%%
%%
\renewcommand{\thesisTitle}{Тема выпускной квалификационной работы}
%\newcommand{\thesisDegree}{бакалавра}% магистра или специалиста% 
\newcommand{\thesisDegree}{работа бакалавра}% дипломный проект, дипломная работа, магистерская диссертация %c 2020
\newcommand{\thesisTitleEn}{Title of the thesis} %2020
\newcommand{\thesisDeadline}{дд.мм.202X}
\newcommand{\thesisStartDate}{дд.мм.202X}
\newcommand{\thesisYear}{202X}
%%
%%
\newcommand{\group}{N} % заменить вместо N номер группы
\newcommand{\thesisSpecialtyCode}{ХХ.ХХ.ХХ}% код направления подготовки
\newcommand{\thesisSpecialtyTitle}{Наименование направления/специальности} % наименование направления/специальности
\newcommand{\thesisOPPostfix}{YY} % последние цифры кода образовательной программы (после <<_>>)
\newcommand{\thesisOPTitle}{Наименование образовательной программы}% наименование образовательной программы
%%
%%
\newcommand{\institute}{
Название института
%Институт компьютерных наук и технологий
%Гуманитарный институт
%Инженерно-строительный институт
%Институт биомедицинских систем и технологий
%Институт металлургии, машиностроения и транспорта
%Институт передовых производственных технологий
%Институт прикладной математики и механики
%Институт физики, нанотехнологий и телекоммуникаций
%Институт физической культуры, спорта и туризма
%Институт энергетики и транспортных систем
%Институт промышленного менеджмента, экономики и торговли
}%
%%
%%




%%% Задание ключевых слов и аннотации
%%
%%
%% Ключевых слов от 3 до 5 слов или словосочетаний в именительном падеже именительном падеже множественного числа (или в единственном числе, если нет другой формы) по правилам русского языка!!!
%%
%%
\newcommand{\keywordsRu}{Стилевое оформление сайта, управление контентом, php, MySQL, архитектура системы} % ВВЕДИТЕ ключевые слова по-русски
%%
%%
\newcommand{\keywordsEn}{Style registration, content management, php, MySQL, system architecture} % ВВЕДИТЕ ключевые слова по-английски
%%
%%
%% Реферат не более 600 знаков на русский или английский текст
\newcommand{\abstractRu}{В данной работе изложена сущность подхода к созданию динамического информационного портала на основе использования открытых технологий Apache, MySQL и PHP. Даны общие понятия и классификация IT-систем такого класса. Проведен анализ систем-прототипов. Изучена технология создания указанного класса информационных систем. Разработана конкретная программная реализация динамического информационного портала на примере портала выбранной тематики.} % ВВЕДИТЕ текст аннотации по-русски
%%
%%
\newcommand{\abstractEn}{In the given work the essence of the approach to creation of a dynamic information portal on the basis of use of open technologies Apache, MySQL and PHP is stated. The general concepts and classification of IT-systems of such class are given. The analysis of systems-prototypes is lead. The technology of creation of the specified class of information systems is investigated. Concrete program realization of a dynamic information portal on an example of a portal of the chosen subjects is developed.} % ВВЕДИТЕ текст аннотации по-английски




%%% Не меняем дальнейшую часть - Do not modify the rest part
%%
%%
%%
%%
\newcommand{\HeadTitle}{\HeadDegree~\HeadDep}
\newcommand{\thesisOPCode}{\thesisSpecialtyCode\_\thesisOPPostfix}% код образовательной программы
\newcommand{\thesisSpecialtyCodeAndTitle}{\thesisSpecialtyCode~\thesisSpecialtyTitle}% Код и наименование направления/специальности
\newcommand{\thesisOPCodeAndTitle}{\thesisOPCode~\thesisOPTitle} % код и наименование образовательной программы
%%
%%
\hypersetup{%часть болка hypesetup в style
		pdftitle={\thesisTitle},    % Заголовок pdf-файла
		pdfauthor={\AuthorFull},    % Автор
		pdfsubject={Выпускная квалификационная работа \thesisDegree. Шифр и наименование направления подготовки: \thesisSpecialtyCodeAndTitle. \abstractRu},      % Тема
		pdfcreator={LaTeX, SPbPU-student-thesis-template},     % Приложение-создатель
%		pdfproducer={},  % Производитель, Производитель PDF % будет выставлена автоматически
		pdfkeywords={\keywordsRu}
}
%%
%%
%% вспомогательные команды
\newcommand{\firef}[1]{рис.\ref{#1}} %figure reference
\newcommand{\taref}[1]{табл.\ref{#1}}	%table reference
%%
%%
%% Архивный вариант задания ключевых слов, аннотации и благодарностей 
% Too hard to export data from the environment to pdf-info
% https://tex.stackexchange.com/questions/184503/collecting-contents-of-environment-and-store-them-for-later-retrieval
%заменить NewEnviron на newenvironment для распознавания команды в TexStudio
%\NewEnviron{keywordsRu}{\noindent\MakeUppercase{\BODY}}
%\NewEnviron{keywordsEn}{\noindent\MakeUppercase{\BODY}}
%\newenvironment{abstractRu}{}{}
%\newenvironment{abstractEn}{}{}
%\newenvironment{acknowledgementsRu}{\par{\normalfont \acknowledgements.}}{}
%\newenvironment{acknowledgementsEn}{\par{\normalfont \acknowledgementsENG.}}{}


%%% Переопределение именований %%% Не меняем - Do not modify
%\newcommand{\Ministry}{Минобрнауки России} 
\newcommand{\Ministry}{Министерство науки и высшего образования Российской~Федерации} %с 2020
\newcommand{\SPbPU}{Санкт-Петербургский политехнический университет Петра~Великого}
%% Пробел между И. О. не допускается.
\renewcommand{\alsoname}{см. также}
\renewcommand{\seename}{см.}
\renewcommand{\headtoname}{вх.}
\renewcommand{\ccname}{исх.}
\renewcommand{\enclname}{вкл.}
\renewcommand{\pagename}{Pages}
\renewcommand{\partname}{Часть}
\renewcommand{\abstractname}{\textbf{Аннотация}}
\newcommand{\abstractnameENG}{\textbf{Annotation}}
\newcommand{\keywords}{\textbf{Ключевые слова}}
\newcommand{\keywordsENG}{\textbf{Keywords}}
\newcommand{\acknowledgements}{\textbf{Благодарности}}
\newcommand{\acknowledgementsENG}{\textbf{Acknowledgements}}
\renewcommand{\contentsname}{Content} % 
%\renewcommand{\contentsname}{Содержание} % (ГОСТ Р 7.0.11-2011, 4)
%\renewcommand{\contentsname}{Оглавление} % (ГОСТ Р 7.0.11-2011, 4)
\renewcommand{\figurename}{Рис.} % Стиль СПбПУ
%\renewcommand{\figurename}{Рисунок} % (ГОСТ Р 7.0.11-2011, 5.3.9)
\renewcommand{\tablename}{Таблица} % (ГОСТ Р 7.0.11-2011, 5.3.10)
%\renewcommand{\indexname}{Предметный указатель}
\renewcommand{\listfigurename}{Список рисунков}
\renewcommand{\listtablename}{Список таблиц}
\renewcommand{\refname}{\fullbibtitle}
\renewcommand{\bibname}{\fullbibtitle}

\newcommand{\chapterEnTitle}{Сhapter title} % <- input the English title here (only once!) 
\newcommand{\chapterRuTitle}{Название главы}          % <- введите 
\newcommand{\sectionEnTitle}{Section title} %<- input subparagraph title in english
\newcommand{\sectionRuTitle}{Название подраздела} % <- введите название подраздела по-русски
\newcommand{\subsectionEnTitle}{Subsection title} % - input subsection title in english
\newcommand{\subsectionRuTitle}{Название параграфа} % <- введите название параграфа по-русски
\newcommand{\subsubsectionEnTitle}{Subsubsection title} % <- input subparagraph title in english
\newcommand{\subsubsectionRuTitle}{Название подпараграфа} % <- введите название подпараграфа по-русски